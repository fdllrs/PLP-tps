\documentclass[10pt,a4paper]{article}

\usepackage[spanish,activeacute,es-tabla]{babel}
\usepackage[utf8]{inputenc}
\usepackage{ifthen}
\usepackage{listings}
\usepackage{dsfont}
\usepackage{subcaption}
\usepackage{amsmath}
\usepackage[strict]{changepage}
\usepackage[top=1cm,bottom=2cm,left=1cm,right=1cm]{geometry}%
\usepackage{color}%
\newcommand{\tocarEspacios}{%
	\addtolength{\leftskip}{3em}%
	\setlength{\parindent}{0em}%
}

% Especificacion de procs

\newcommand{\In}{\textsf{in }}
\newcommand{\Out}{\textsf{out }}
\newcommand{\Inout}{\textsf{inout }}

\newcommand{\encabezadoDeProc}[4]{%
	% Ponemos la palabrita problema en tt
	%  \noindent%
	{\normalfont\bfseries\ttfamily proc}%
	% Ponemos el nombre del problema
	\ %
	{\normalfont\ttfamily #2}%
	\
	% Ponemos los parametros
	(#3)%
	\ifthenelse{\equal{#4}{}}{}{%
		% Por ultimo, va el tipo del resultado
		\ : #4}
}

\newenvironment{proc}[4][res]{%
	
	% El parametro 1 (opcional) es el nombre del resultado
	% El parametro 2 es el nombre del problema
	% El parametro 3 son los parametros
	% El parametro 4 es el tipo del resultado
	% Preambulo del ambiente problema
	% Tenemos que definir los comandos requiere, asegura, modifica y aux
	\newcommand{\requiere}[2][]{%
		{\normalfont\bfseries\ttfamily requiere}%
		\ifthenelse{\equal{##1}{}}{}{\ {\normalfont\ttfamily ##1} :}\ %
		\{\ensuremath{##2}\}%
		{\normalfont\bfseries\,\par}%
	}
	\newcommand{\asegura}[2][]{%
		{\normalfont\bfseries\ttfamily asegura}%
		\ifthenelse{\equal{##1}{}}{}{\ {\normalfont\ttfamily ##1} :}\
		\{\ensuremath{##2}\}%
		{\normalfont\bfseries\,\par}%
	}
	\renewcommand{\aux}[4]{%
		{\normalfont\bfseries\ttfamily aux\ }%
		{\normalfont\ttfamily ##1}%
		\ifthenelse{\equal{##2}{}}{}{\ (##2)}\ : ##3\, = \ensuremath{##4}%
		{\normalfont\bfseries\,;\par}%
	}
	\renewcommand{\pred}[3]{%
		{\normalfont\bfseries\ttfamily pred }%
		{\normalfont\ttfamily ##1}%
		\ifthenelse{\equal{##2}{}}{}{\ (##2) }%
		\{%
		\begin{adjustwidth}{+5em}{}
			\ensuremath{##3}
		\end{adjustwidth}
		\}%
		{\normalfont\bfseries\,\par}%
	}
	
	\newcommand{\res}{#1}
	\vspace{1ex}
	\noindent
	\encabezadoDeProc{#1}{#2}{#3}{#4}
	% Abrimos la llave
	\par%
	\tocarEspacios
}
{
	% Cerramos la llave
	\vspace{1ex}
}

\newcommand{\aux}[4]{%
	{\normalfont\bfseries\ttfamily\noindent aux\ }%
	{\normalfont\ttfamily #1}%
	\ifthenelse{\equal{#2}{}}{}{\ (#2)}\ : #3\, = \ensuremath{#4}%
	{\normalfont\bfseries\,;\par}%
}

\newcommand{\pred}[3]{%
	{\normalfont\bfseries\ttfamily\noindent pred }%
	{\normalfont\ttfamily #1}%
	\ifthenelse{\equal{#2}{}}{}{\ (#2) }%
	\{%
	\begin{adjustwidth}{+2em}{}
		\ensuremath{#3}
	\end{adjustwidth}
	\}%
	{\normalfont\bfseries\,\par}%
}

% Tipos

\newcommand{\nat}{\ensuremath{\mathds{N}}}
\newcommand{\ent}{\ensuremath{\mathds{Z}}}
\newcommand{\float}{\ensuremath{\mathds{R}}}
\newcommand{\bool}{\ensuremath{\mathsf{Bool}}}
\newcommand{\cha}{\ensuremath{\mathsf{Char}}}
\newcommand{\str}{\ensuremath{\mathsf{String}}}

% Logica

\newcommand{\True}{\ensuremath{\mathrm{true}}}
\newcommand{\False}{\ensuremath{\mathrm{false}}}
\newcommand{\Then}{\ensuremath{\rightarrow}}
\newcommand{\Iff}{\ensuremath{\leftrightarrow}}
\newcommand{\implica}{\ensuremath{\longrightarrow}}
\newcommand{\IfThenElse}[3]{\ensuremath{\mathsf{if}\ #1\ \mathsf{then}\ #2\ \mathsf{else}\ #3\ \mathsf{fi}}}
\newcommand{\yLuego}{\land _L}
\newcommand{\oLuego}{\lor _L}
\newcommand{\implicaLuego}{\implica _L}

\newcommand{\cuantificador}[5]{%
	\ensuremath{(#2 #3: #4)\ (%
		\ifthenelse{\equal{#1}{unalinea}}{
			#5
		}{
			$ % exiting math mode
			\begin{adjustwidth}{+2em}{}
				$#5$%
			\end{adjustwidth}%
			$ % entering math mode
		}
		)}
}

\newcommand{\existe}[4][]{%
	\cuantificador{#1}{\exists}{#2}{#3}{#4}
}
\newcommand{\paraTodo}[4][]{%
	\cuantificador{#1}{\forall}{#2}{#3}{#4}
}

%listas

\newcommand{\TLista}[1]{\ensuremath{seq \langle #1\rangle}}
\newcommand{\lvacia}{\ensuremath{[\ ]}}
\newcommand{\lv}{\ensuremath{[\ ]}}
\newcommand{\longitud}[1]{\ensuremath{|#1|}}
\newcommand{\cons}[1]{\ensuremath{\mathsf{addFirst}}(#1)}
\newcommand{\indice}[1]{\ensuremath{\mathsf{indice}}(#1)}
\newcommand{\conc}[1]{\ensuremath{\mathsf{concat}}(#1)}
\newcommand{\cab}[1]{\ensuremath{\mathsf{head}}(#1)}
\newcommand{\cola}[1]{\ensuremath{\mathsf{tail}}(#1)}
\newcommand{\sub}[1]{\ensuremath{\mathsf{subseq}}(#1)}
\newcommand{\en}[1]{\ensuremath{\mathsf{en}}(#1)}
\newcommand{\cuenta}[2]{\mathsf{cuenta}\ensuremath{(#1, #2)}}
\newcommand{\suma}[1]{\mathsf{suma}(#1)}
\newcommand{\twodots}{\ensuremath{\mathrm{..}}}
\newcommand{\masmas}{\ensuremath{++}}
\newcommand{\matriz}[1]{\TLista{\TLista{#1}}}
\newcommand{\seqchar}{\TLista{\cha}}

\renewcommand{\lstlistingname}{Código}
\lstset{% general command to set parameter(s)
	language=Java,
	morekeywords={endif, endwhile, skip},
	basewidth={0.47em,0.40em},
	columns=fixed, fontadjust, resetmargins, xrightmargin=5pt, xleftmargin=15pt,
	flexiblecolumns=false, tabsize=4, breaklines, breakatwhitespace=false, extendedchars=true,
	numbers=left, numberstyle=\tiny, stepnumber=1, numbersep=9pt,
	frame=l, framesep=3pt,
	captionpos=b,
}

\usepackage{caratula} % Version modificada para usar las macros de algo1 de ~> https://github.com/bcardiff/dc-tex
\usepackage[dvipsnames]{xcolor}
\usepackage{amsmath}

\titulo{Trabajo práctico 1}
\subtitulo{Programación Funcional}

\fecha{\today}

\materia{Paradigmas de Programación}
\grupo{Grupo CHAD sociedad anónima}


\integrante{Condori Llanos, Alex}{163/23}{nocwe11@gmail.com}
\integrante{Della Rosa, Facundo César}{1317/23}{dellarosafacundo@gmail.com}
\integrante{López Porto, Gregorio}{1376/23}{gregoriolopezporto@gmail.com}
\integrante{Winogron, Iván}{459/23}{Ivowino2000@gmail.com}
\begin{document}


\maketitle


\section*{Ejercicio 9}
\subsection*{Enunciado}
De acuerdo a las definiciones de las funciones para árboles ternarios de más arriba, se pide
demostrar lo siguiente: \\
\begin{equation*}
	\forall t :: \text{AT a . }\forall x :: \text{a . (elem x (preorder t) = elem x (postorder t))} 
\end{equation*}

\subsection*{Definiciones}
\noindent
$elem$ :: Eq a  $\implies$ a $\rightarrow$ [a] $\rightarrow$ Bool \\
\{E0\} elem e [ ] = False \\
\{E1\} elem e (x:xs) = (e == x) $||$ elem e xs \\
\\
$preorder$ :: Procesador (AT a) a \\
\{PRE1\} preorder = foldAT ($\backslash$x ri rc rd $\rightarrow$ concat [[x], ri, rc, rd]) [ ]\\
\\
$postorder$ :: Procesador (AT a) a \\
\{POST1\} postorder = foldAT ($\backslash$x ri rc rd $\rightarrow$ concat [ri, rc, rd, [x]]) [ ]\\
\\
$foldAT :: (a \rightarrow b \rightarrow b \rightarrow b \rightarrow b) \rightarrow b \rightarrow AT a \rightarrow b $\\
\{F0\} foldAT f b Nil = b\\
\{F1\} foldAT f b (Tern a ri rc rd) = f a (foldAT f b ri) (foldAT f b rc) (foldAT f b rd) \\

\subsection*{Demostración (esqueleto, faltaría formalizar y emprolijar)}
\noindent
Por inducción estructural en t\\
P(t) = elem x (preorder t) = elem x (postorder t) \\ \\
Caso base: P(Nil) = elem x (preorder Nil) = elem x (postorder Nil)\\ \\
elem x (preorder Nil) $\underset{\{PRE1\}}{=}$ elem x (foldAT ($\backslash$x ri rc rd $\rightarrow$ x : concat [ri, rc, rd]) [ ] Nil) $\underset{\{F0\}}{=}$ elem x [ ] \\

\noindent análogamente: \\
elem x (postorder Nil) $\underset{\{POST1\}}{=}$ elem x (foldAT ($\backslash$x ri rc rd $\rightarrow$ concat [ri, rc, rd, [x]]) [ ] Nil) $\underset{\{F0\}}{=}$ elem x [ ] \\


Luego vale el caso base P(Nil)
\\ \\
Paso inductivo: 
\begin{center}
	$\forall$ h1 :: AT a, $\forall$h2 :: AT a, $\forall$h3 :: AT a, $\forall$r :: a, \\
	P(h1) $\land$ P(h2) $\land$ P(h3) $\land$ $\implies$ P(Tern a h1 h2 h3)
\end{center}
Es decir, supongo que valen P(h1), P(h2), P(h3) y quiero ver que vale P(Tern a h1 h2 h3) \\
P(h1) = elem x (preorder h1) = elem x (postorder h1) \\
P(h2) = elem x (preorder h2) = elem x (postorder h2) \\
P(h3) = elem x (preorder h3) = elem x (postorder h3) \\
P(Tern a h1 h2 h3) = elem x (preorder (Tern a h1 h2 h3)) = elem x (postorder (Tern a h1 h2 h3)) \\
\\
\noindent

elem x (postorder (Tern a h1 h2 h3)) $\underset{\{POST1\}}{=}$ elem x (foldAT ($\backslash$x r1 rc rd $\rightarrow$ concat [ri, rc, rd, [x]]) []) (Tern a h1 h2 h3)
considero f = ($\backslash$x r1 rc rd $\rightarrow$ concat [r1, rc, rd, [x]]) para facilitar la lectura.\\ \\
$\underset{\{F1\}}{=}$ elem x ((f a (foldAT f [] r1) (foldAT f [] rc) (foldAT f [] rd)) (Tern a h1 h2 h3)) \\
$\underset{aplico}{=}$ elem x (($\backslash$x r1 rc rd $\rightarrow$ concat [r1, rc, rd, [x]]) a (foldAT f [] h1) (foldAT f [] h2) (foldAT f [] h3)) \\
$\underset{aplico}{=}$ elem x (concat [(foldAT f [] h1), (foldAT f [] h2), (foldAT f [] h3), [a]]) \\

utilizando el siguiente lema : elem x (concat [a,b,c,d]) = elem x a $||$ elem x b $||$ elem x c $||$ elem x d \\
\\
\noindent
$\underset{\{lema\}}{=}$ elem x (foldAT f [] h1) $||$ elem x (foldAT f [] h2) $||$ elem x (foldAT f [] h3) $||$ elem x [a] \\
$\underset{\{POST1\}}{=}$ elem x (postorder h1) $||$ elem x (postorder h2) $||$ elem x (postorder h3) $||$ elem x [a] \\
$\underset{\{HI's\}}{=}$ elem x (preorder h1) $||$ elem x (preorder h2) $||$ elem x (preorder h3) $||$ elem x [a] \\
reordeno los términos \\
= elem x [a] $||$ elem x (preorder h1) $||$ elem x (preorder h2) $||$ elem x (preorder h3)  \\
$\underset{\{lema\}}{=}$ elem x concat [[a], (preorder h1), (preorder h2), (preorder h3)]  \\
$\underset{\{des-aplico\}}{=}$ elem x (foldAT ($\backslash$x ri rc rd $\rightarrow$ concat [[x], (preorder ri), (preorder rc), (preorder rd)]) (Tern a h1 h2 h3))  \\
$\underset{\{PRE1\}}{=}$ elem x (preorder (Tern a h1 h2 h3))  \\
\\
Entonces, mediante una cadena de igualdades, concluyo que: elem x (preorder (Tern a h1 h2 h3)) = elem x (postorder (Tern a h1 h2 h3)), que es lo que quería probar.

\end{document}
