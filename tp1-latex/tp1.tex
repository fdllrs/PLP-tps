\documentclass[10pt,a4paper]{article}

\input{AEDmacros}
\usepackage{caratula} % Version modificada para usar las macros de algo1 de ~> https://github.com/bcardiff/dc-tex
\usepackage[dvipsnames]{xcolor}
\usepackage{amsmath}
\linespread{1.3}
\titulo{Trabajo práctico 1}
\subtitulo{Programación Funcional}

\fecha{\today}

\materia{Paradigmas de Programación}
\grupo{Grupo CHAD sociedad anónima}


\integrante{Condori Llanos, Alex}{163/23}{nocwe11@gmail.com}
\integrante{Della Rosa, Facundo César}{1317/23}{dellarosafacundo@gmail.com}
\integrante{López Porto, Gregorio}{1376/23}{gregoriolopezporto@gmail.com}
\integrante{Winogron, Iván}{459/23}{Ivowino2000@gmail.com}
\begin{document}


\maketitle


\section*{Ejercicio 9}
\subsection*{Enunciado}
De acuerdo a las definiciones de las funciones para árboles ternarios de más arriba, se pide
demostrar lo siguiente:
\begin{equation*}
	\forall t :: \text{AT a . }\forall x :: \text{a . (elem x (preorder t) = elem x (postorder t))} 
\end{equation*}

\subsection*{Definiciones}
\noindent
$elem$ :: Eq a  $\implies$ a $\rightarrow$ [a] $\rightarrow$ Bool \\
\{E0\} elem e [ ] = False \\
\{E1\} elem e (x:xs) = (e == x) $||$ elem e xs \\
\\
$preorder$ :: Procesador (AT a) a \\
\{PRE1\} preorder = foldAT ($\backslash$x ri rc rd $\rightarrow$ [x] ++ ri ++ rc ++ rd) [ ]\\
\\
$postorder$ :: Procesador (AT a) a \\
\{POST1\} postorder = foldAT ($\backslash$x ri rc rd $\rightarrow$ ri ++ rc ++ rd ++ [x]) [ ]\\
\\
$foldAT :: (a \rightarrow b \rightarrow b \rightarrow b \rightarrow b) \rightarrow b \rightarrow AT a \rightarrow b $\\
\{F0\} foldAT f b Nil = b\\
\{F1\} foldAT f b (Tern r ri rc rd) = f a (foldAT f b ri) (foldAT f b rc) (foldAT f b rd)

\subsection*{Demostración}
Hacemos la demostración utilizando inducción estructural en el arbol ternario t. Sea P(t) la propiedad:
\begin{center}
	P(t) = elem x (preorder t) = elem x (postorder t)
\end{center}
Primero pruebo la propiedad para el caso base: 
\begin{center}
	P(Nil) = elem x (preorder Nil) = elem x (postorder Nil)
\end{center}
elem x (preorder Nil) $\underset{\{PRE1\}}{=}$ elem x (foldAT ($\backslash$x ri rc rd $\rightarrow$ [x] ++ ri ++ rc ++ rd) [ ] Nil) $\underset{\{F0\}}{=}$ elem x [ ] \\
elem x (postorder Nil) $\underset{\{POST1\}}{=}$ elem x (foldAT ($\backslash$x ri rc rd $\rightarrow$ concat ri ++ rc ++ rd ++ [x]) [ ] Nil) $\underset{\{F0\}}{=}$ elem x [ ] \\

\noindent
Entonces elem x (preorder Nil) = elem x (postorder Nil) = elem x [ ]. Luego vale la propiedad para el caso base P(Nil).
\\
Pruebo el paso inductivo: 
\begin{center}
	$\forall$ h1 :: AT a, $\forall$h2 :: AT a, $\forall$h3 :: AT a, $\forall$r :: a, \\
	P(h1) $\land$ P(h2) $\land$ P(h3) $\land$ $\implies$ P(Tern r h1 h2 h3)
\end{center}
Defino las propiedades \\
P(h1) = elem x (preorder h1) = elem x (postorder h1) \\
P(h2) = elem x (preorder h2) = elem x (postorder h2) \\
P(h3) = elem x (preorder h3) = elem x (postorder h3) \\
P(Tern r h1 h2 h3) = elem x (preorder (Tern r h1 h2 h3)) = elem x (postorder (Tern r h1 h2 h3)) \\
\noindent
Entonces, mi hipótesis inductiva es que valen las propiedades P(h1), P(h2), P(h3), y quiero ver que vale P(Tern r h1 h2 h3). \\

\noindent
elem x (postorder (Tern r h1 h2 h3)) $\underset{\{POST1\}}{=}$ elem x (foldAT ($\backslash$x ri rc rd $\rightarrow$ ri ++ rc ++ rd ++ [x]) [ ]) (Tern r h1 h2 h3) \\
\\
considero f = ($\backslash$x ri rc rd $\rightarrow$ ri ++ rc ++ rd ++ [x]) para facilitar la lectura.\\ 
$\underset{\{F1\}}{=}$ elem x ((f a (foldAT f [ ] ri) (foldAT f [ ] rc) (foldAT f [ ] rd)) (Tern r h1 h2 h3)) \\
$\underset{\beta\rightarrow}{=}$ elem x (($\backslash$x ri rc rd $\rightarrow$ ri ++ rc ++ rd ++ [x]) r (foldAT f [ ] h1) (foldAT f [ ] h2) (foldAT f [ ] h3))\\
$\underset{\beta\rightarrow}{=}$ elem x ((foldAT f [ ] h1) ++ (foldAT f [ ] h2) ++ (foldAT f [ ] h3) ++ [r])\\

\noindent
Propongo el siguiente lema: 
\begin{center}
	$\forall$ a :: [z], $\forall$b :: [z], $\forall$c :: [z], $\forall$d :: [z], \\
	elem x (a ++ b ++ c ++ d) = elem x a $||$ elem x b $||$ elem x c $||$ elem x d
\end{center}
\noindent
considero g = ($\backslash$x ri rc rd $\rightarrow$ [x] ++ ri ++ rc ++ rd) para facilitar la lectura.\\ 
$\underset{\{lema\}}{=}$ elem x (foldAT f [ ] h1) $||$ elem x (foldAT f [ ] h2) $||$ elem x (foldAT f [ ] h3) $||$ elem x [r] \\
$\underset{\{POST1\}}{=}$ elem x (postorder h1) $||$ elem x (postorder h2) $||$ elem x (postorder h3) $||$ elem x [r] \\
$\underset{\{HI's\}}{=}$ elem x (preorder h1) $||$ elem x (preorder h2) $||$ elem x (preorder h3) $||$ elem x [r] \\
\\
Utilizando la conmutatividad de la disyunción, cambio de lugar los términos \\
= elem x [r] $||$ elem x (preorder h1) $||$ elem x (preorder h2) $||$ elem x (preorder h3)  \\
$\underset{\{lema\}}{=}$ elem x ([r] ++ (preorder h1) ++ (preorder h2) ++ (preorder h3))  \\
$\underset{\{\beta \leftarrow\}}{=}$ elem x (($\backslash$x ri rc rd $\rightarrow$ ([x] ++ ri ++ rc ++ rd)) r (foldAT g [ ] h1) (foldAT g [ ] h2) (foldAT g [ ] h3))  \\
$\underset{\{\beta \leftarrow\}}{=}$ elem x ((g a (foldAT g [ ] ri) (foldAT g [ ] rc) (foldAT g [ ] rd)) (Tern r h1 h2 h3))  \\
$\underset{\{PRE1\}}{=}$ elem x (preorder (Tern r h1 h2 h3))  \\
Entonces, mediante una cadena de igualdades, concluyo que elem x (preorder (Tern r h1 h2 h3)) = elem x (postorder (Tern r h1 h2 h3)), que es lo que quería probar.
\\ \\
\noindent
Demostración del lema elem x (a ++ b ++ c ++ d) = elem x a $||$ elem x b $||$ elem x c $||$ elem x d \\
voy a demostrar primero una versión simplificada para luego demostrar el lema. \\
llamo Q a la propiedad: elem x ($l_1$ ++ $l_2$) = elem x $l_1$ $||$ elem x $l_2$ \\
Para demostrar que las funciones son iguales, basta con ver que 
\begin{center}
	$\forall$ $l_1$ :: [a], $\forall$$l_2$ :: [a] . elem x ($l_1$ ++ $l_2$) = elem x $l_1$ $||$ elem x $l_2$
\end{center}
Para la demostración voy a hacer inducción estructural sobre la lista $l_1$. \\
caso base Q([]) = elem x ([] ++ $l_2$) $\underset{\{lista\}}{=}$ elem x $l_2$. \\
por otro lado, elem x [] $||$ elem x $l_2$ $\underset{\{E0\}}{=}$ False $||$ elem x $l_2$ = elem x $l_2$ \\
luego vale la propiedad Q para el caso base. Veo ahora el paso inductivo
\begin{center}
	$\forall$ ys :: [a], $\forall$y :: a . Q(ys) $\implies$ Q(y:ys)
\end{center}
defino las propiedades \\
Q(ys) = elem x (ys ++ $l_2$) = elem x ys $||$ elem x $l_2$ \\
Q(y:ys) = elem x ((y:ys) ++ $l_2$) = elem x y:ys $||$ elem x $l_2$ \\ \\
Entonces, mi hipótesis inductiva es que vale la propiedad Q(ys) y mi objetivo es ver que vale Q(y:ys). \\
elem x ((y:ys) ++ $l_2$) $\underset{\{E1\}}{=}$ x == y $||$ elem x (ys ++ $l_2$) $\underset{\{HI\}}{=}$ x == y $||$ elem x ys $||$ elem x $l_2$ \\
= (x == y $||$ elem x ys) $||$ elem x $l_2$ $\underset{\{E1\}}{=}$ elem x (y:ys) $||$ elem x $l_2$ \\
entonces vale la propiedad Q. \\
Ahora para demostrar el lema, considero $l_1$ = a, $l_2$ = (b ++ c ++ d) y aplico la propiedad Q: \\
elem x (a ++ b ++ c ++ d) $\underset{\{Q\}}{=}$ elem x a $||$ elem x (b ++ c ++ d) \\
Siguiendo la misma lógica, considero $l_1$ = b, $l_2$ = (c ++ d) y aplico nuevamente la propiedad Q sobre el segundo término: \\
elem x a $||$ elem x (b ++ c ++ d) $\underset{\{Q\}}{=}$ elem x a $||$ elem x b $||$ elem x (c ++ d) \\
Aplico la propeidad Q nuevamente considerando $l_1$ = c, $l_2$ = d: \\
elem x a $||$ elem x b $||$ elem x (c ++ d) $\underset{\{Q\}}{=}$ elem x a $||$ elem x b $||$ elem x c $||$ elem x d \\
Luego, queda demostrado el lema elem x (a ++ b ++ c ++ d) = elem x a $||$ elem x b $||$ elem x c $||$ elem x d.

\end{document}
